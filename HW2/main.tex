
%hi
\documentclass{article}
\usepackage[margin=1in]{geometry} 
\usepackage{amsmath,amsthm,amssymb,amsfonts, fancyhdr, color, comment, graphicx, environ}
\usepackage{xcolor}
\usepackage{mdframed}
\usepackage[shortlabels]{enumitem}
\usepackage{indentfirst}
\usepackage{hyperref}
\renewcommand{\footrulewidth}{0.8pt}
\hypersetup{
    colorlinks=true,
    linkcolor=blue,
    filecolor=magenta,      
    urlcolor=blue,
}


\pagestyle{fancy}

\DeclareRobustCommand{\bbone}{\text{\usefont{U}{bbold}{m}{n}1}}

\DeclareMathOperator{\EX}{\mathbb{E}}% expected value


\newenvironment{problem}[2][Problem]
    { \begin{mdframed}[backgroundcolor=gray!20] \textbf{#1 #2} \\}
    {  \end{mdframed}}


\newenvironment{solution}{\textbf{Solution}}


\lhead{Kasra Amani}
\rhead{Computer Simulation} 
\chead{\textbf{Assignment 2}}
\lfoot{Dr. Bardia Safaei}
\rfoot{Sharif University of Technology}
\def\thesection{\alph{section}}

\begin{document}
\input{coverPage}%not necessary but looks fancy
    \begin{problem}{1}
    	\begin{section}{}
    		\noindent
    		$1 - P(X\leq x) = P(X>x)$ \\
    		We want to know how many $P(x_i)$s are in this summation; it is repeated in every 
    		sequence until we reach $x=x_i$ and thus, it is repeated $x_i$ times; so we can 
    		rewrite the given statement as follows:\\
    		$\displaystyle\sum_{x=0}^\infty x_iP(X=x_i) = \EX(X)$
    	\end{section}
    	
    	\begin{section}{}
    		\noindent
    		We need to show that $P(Z=z)=\frac{e^{-\lambda}\lambda^z}{z!}$ where 
    		$\lambda = \lambda_1+\lambda_2$:\\
    		Since $\displaystyle Z=X+Y$: $P(Z=z) = \displaystyle\sum_{i=0}^{z}P(X=i,Y=z-i)$; $X$ and $Y$ are independent 
    		thus:\\ 
    		$=\displaystyle\sum_{i=0}^{z}P(X=i)P(Y=z-i) = \sum_{i=0}^z\frac{1}{i!(z-i)!}e^{-						\lambda_1}\lambda_1^ie^{-\lambda_2}\lambda_2^{z-i} = \sum_{i=0}^z\frac{z!}{i!(z-i)!}					\frac{e^{-\lambda_1}\lambda_1^ie^{-\lambda_2}\lambda_2^{z-i}}{z!}$ \\
    		$= \displaystyle\sum_{i=0}^{z}\displaystyle\binom{z}{i}\frac{e^{-\lambda_1}								\lambda_1^ie^{-\lambda_2}\lambda_2^{z-i}}				{z!} = \displaystyle\sum_{i=0}					^{z}\displaystyle\binom{z}{i}\frac{e^{-(\lambda_1+						\lambda_2)}}{z!}				\lambda_1^i\lambda_2^{z-i} = \frac{e^{-\lambda}}{z!}\displaystyle\sum_{i=0}^{z}							\displaystyle\binom{z}{i}\lambda_1^i\lambda_2^{z-i}$\\
    		(using binomial expansion) $=  \displaystyle\frac{e^{-\lambda}}{z!}(\lambda_1+							\lambda_2)^z = 
    		\frac{e^{-\lambda}\lambda^z}{z!}$ and the proof is concluded.
    		
    	\end{section}
    \end{problem}
    
    \begin{problem}{2}
    	\\
    	\textbf{Lemma:} The $n$th person necessarily sits in either the first seat or the
    	last one. \\ \textbf{Proof by contradiction:} If the $n$th person isn't sitting in 
    	either of those two, he must be sitting in the $i$th seat where $2\leq i\leq n-1$;
    	so the $i$th seat was empty when the last person arrived and decided to sit in it;
    	but if it had been empty until the end, it was also empty the moment that $i$th person
    	himself arrived and he must have sat there but he didn't and the $n$th person is now
    	sitting there so the lemma must be correct. \\ \\
    	We now divide the problem into $n$ steps; each step consists of the $i$th person arriving 
    	and taking a seat. The fate of the last person is decided the moment someone sits
    	in either the last or the first seat (because of the lemma proven earlier) and since 
    	the chances of the last or first being chosen is equal in all of the $n$ 
    	steps (either $0$ when the person deciding to sit down has their designated seat available
    	or equal to each other otherwise), the chances of the last person arriving and finding 
    	their seat empty is $\displaystyle\frac{p_1}{2}+\frac{p_2}{2}+...+\frac{p_n}{2} = \frac{p_1+p_2+...+p_n}{2}$   where we know $\displaystyle\sum_{i=1}^{n}p_i=1$ and thus, the chances of the last person
    	finding their seat empty upon arrival is $\frac{1}{2}$.
    \end{problem}
    
    \begin{problem}{3}
    	\textbf{Note:} If the processing time for a task is less than or equal to zero, 
    	the simulator assigns $0.01$ to the variable since a task being finished the moment 
    	it enters the queue is kinda dumb :))
    \end{problem}
    
    \begin{problem}{4}
    	TBD.
    \end{problem}
    
    \begin{problem}{5}
    	Code(TBD).
    \end{problem}
    
\end{document}