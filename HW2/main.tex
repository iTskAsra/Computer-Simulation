
%hi
\documentclass{article}
\usepackage[margin=1in]{geometry} 
\usepackage{amsmath,amsthm,amssymb,amsfonts, fancyhdr, color, comment, graphicx, environ}
\usepackage{xcolor}
\usepackage{mdframed}
\usepackage[shortlabels]{enumitem}
\usepackage{indentfirst}
\usepackage{hyperref}
\renewcommand{\footrulewidth}{0.8pt}
\hypersetup{
    colorlinks=true,
    linkcolor=blue,
    filecolor=magenta,      
    urlcolor=blue,
}


\pagestyle{fancy}

\DeclareRobustCommand{\bbone}{\text{\usefont{U}{bbold}{m}{n}1}}

\DeclareMathOperator{\EX}{\mathbb{E}}% expected value


\newenvironment{problem}[2][Problem]
    { \begin{mdframed}[backgroundcolor=gray!20] \textbf{#1 #2} \\}
    {  \end{mdframed}}


\newenvironment{solution}{\textbf{Solution}}


\lhead{Kasra Amani}
\rhead{Computer Simulation} 
\chead{\textbf{Assignment 2}}
\lfoot{Dr. Bardia Safaei}
\rfoot{Sharif University of Technology}
\def\thesection{\alph{section}}

\begin{document}
\input{coverPage}%not necessary but looks fancy
    \begin{problem}{1}
    	\begin{section}{}
    		\noindent
    		$1 - P(X\leq x) = P(X>x)$ \\
    		We want to know how many $P(x_i)$s are in this summation; it is repeated in every 
    		sequence until we reach $x=x_i$ and thus, it is repeated $x_i$ times; so we can 
    		rewrite the given statement as follows:\\
    		$\displaystyle\sum_{x=0}^\infty x_iP(X=x_i) = \EX(X)$
    	\end{section}
    	
    	\begin{section}{}
    		\noindent
    		We need to show that $P(Z=z)=\frac{e^{-\lambda}\lambda^z}{z!}$ where 
    		$\lambda = \lambda_1+\lambda_2$:\\
    		Since $\displaystyle Z=X+Y$: $P(Z=z) = \displaystyle\sum_{i=0}^{z}P(X=i,Y=z-i)$; $X$ and $Y$ are independent 
    		thus:\\ 
    		$=\displaystyle\sum_{i=0}^{z}P(X=i)P(Y=z-i) = \sum_{i=0}^z\frac{1}{i!(z-i)!}e^{-						\lambda_1}\lambda_1^ie^{-\lambda_2}\lambda_2^{z-i} = \sum_{i=0}^z\frac{z!}{i!(z-i)!}					\frac{e^{-\lambda_1}\lambda_1^ie^{-\lambda_2}\lambda_2^{z-i}}{z!}$ \\
    		$= \displaystyle\sum_{i=0}^{z}\displaystyle\binom{z}{i}\frac{e^{-\lambda_1}								\lambda_1^ie^{-\lambda_2}\lambda_2^{z-i}}				{z!} = \displaystyle\sum_{i=0}					^{z}\displaystyle\binom{z}{i}\frac{e^{-(\lambda_1+						\lambda_2)}}{z!}				\lambda_1^i\lambda_2^{z-i} = \frac{e^{-\lambda}}{z!}\displaystyle\sum_{i=0}^{z}							\displaystyle\binom{z}{i}\lambda_1^i\lambda_2^{z-i}$\\
    		(using binomial expansion) $=  \displaystyle\frac{e^{-\lambda}}{z!}(\lambda_1+							\lambda_2)^z = 
    		\frac{e^{-\lambda}\lambda^z}{z!}$ and the proof is concluded.
    		
    	\end{section}
    \end{problem}
    
    \begin{problem}{2}
    	
    \end{problem}
    
    \begin{problem}{3}
    	
    \end{problem}
    
    \begin{problem}{4}
    	TBD.
    \end{problem}
    
    \begin{problem}{5}
    	Code.
    \end{problem}
    
\end{document}