
%hi
\documentclass{article}
\usepackage[margin=1in]{geometry} 
\usepackage{amsmath,amsthm,amssymb,amsfonts, fancyhdr, color, comment, graphicx, environ}
\usepackage{xcolor}
\usepackage{mdframed}
\usepackage[shortlabels]{enumitem}
\usepackage{indentfirst}
\usepackage{hyperref}
\renewcommand{\footrulewidth}{0.8pt}
\hypersetup{
    colorlinks=true,
    linkcolor=blue,
    filecolor=magenta,      
    urlcolor=blue,
}


\pagestyle{fancy}

\DeclareRobustCommand{\bbone}{\text{\usefont{U}{bbold}{m}{n}1}}

\DeclareMathOperator{\EX}{\mathbb{E}}% expected value


\newenvironment{problem}[2][Problem]
    { \begin{mdframed}[backgroundcolor=gray!20] \textbf{#1 #2} \\}
    {  \end{mdframed}}


\newenvironment{solution}{\textbf{Solution}}


\lhead{Kasra Amani}
\rhead{Computer Simulation} 
\chead{\textbf{Assignment 2}}
\lfoot{Dr. Bardia Safaei}
\rfoot{Sharif University of Technology}
\def\thesection{\alph{section}}

\begin{document}
\input{coverPage}%not necessary but looks fancy
    \begin{problem}{1}
    	\begin{section}{}
    	\noindent
    	\textbf{1. } Results must be produced quickly. \\
    	\textbf{2. } Different computer architectures must be capable of running the algorithm. \\
    	\textbf{3. } Results must be repeatable. \\
    	\textbf{4. } Imitate the uniformity and independence of an ideal random number generator. \\
    	\textbf{5. } Long life cycles. \\
    	\end{section}
    	\begin{section}{}
    		\noindent
    		\textbf{1. } Since $c\neq0$ it's a case of mixed congruential
    		method and since $m = 2^b$, $c$ is prime to $m$ and $a=1+4k$
    		the cycle length of the algorithm is $m=2^b=256$.
    		\\ \\ \textbf{2. } This is a case of multiplicative congruential methods and with an odd seed($X_0$), since $m=2^{10}$ and 
    		$a=8\times814 + 5$, the cycle will be $\frac{m}{4}=256$.
    	\end{section}
    
    \end{problem}
	\begin{problem}{2}
		Code.
	\end{problem}
    
\end{document}